\documentclass[12pt, a4]{article}

\usepackage{natbib}

\bibliographystyle{abbrvnat}
\setcitestyle{authoryear,open={(},close={)}}

\title{Scrum team reflective journal}
\author{Joshua Lee (G00373792)}
\date{\today}

\begin{document}

\begin{titlepage}
\maketitle
\tableofcontents
\end{titlepage}

\section{Introduction}
This journal is intended to provide a distillation and exploration of the skills
I've learned and developed as part of the project management module, and as part
of an agile team developing a program for the market. This journal will act as
a repository and future reference as I continue to grow as a developer and future
agile team member.

This not only includes communication, team-working and agile-related skills, but
also basic skills such as time-keeping and the discipline to be a reliable and
enthusiastic team member.

Furthermore, this document will prove to be an invaluable guide to avoiding any
technical or social mistakes that, as an inexperienced software engineer, I will
inevitably make over the course of the project.

\section{Journal entries}
\subsection{Week two}
Teams are the centre of an agile workflow and the core task for the first week
was to form a capable team to develop a software project. The focus on
non-technical aspects of a team such as personal interests and hobbies opened
my eyes to the importance of interpersonal relationships to the efficient
functioning of an agile team.

Subsequent to forming our team, ``Starship Troopers'', the next step was to
record its members and brainstorm several ideas of what the project could be.
Interestingly, an agile team does this in a visual way, laying out team members,
their respective interest(s) and a selection of ideas on a large piece of paper
that was then displayed on the wall. I found this approach to be helpful for
visualising the road ahead; it reminded me of my time in junior or
senior infants.

As discussed in the lecture during this week, an agile team is divided into
several different roles:

\begin{itemize}
\item Product owner
\item Scrum master
\item Software engineer 
\item Tester
\end{itemize}

However, I also learned that agile methodologies emphasise fluid roles and
individual initiative over a rigid structure. A team member in any one role
should not isolate themselves to their role title, but should be prepared to
work on a variety of different areas and across disciplines. For example, each
team member should be prepared to test the product even if that is not their
designated role.

To maximise the effectiveness of our team and develop each member's potential
as much as possible, we explored which individual would be most suited to each
of these roles. I assumed the position as one of the software engineers/testers as
I have the strongest track record of writing software and my skills are best
utilised in that role.

Eoin Lernihan took the role of product owner as he has a clear vision of the
high-level goals of such a project, and it was decided unanimously that Kevin
Maloney would fulfill the role of scrum master most effectively. Lastly, David
Newman would become a fellow engineer/tester.

\subsection{Week two}
Our team rapidly converged on an idea for our project: a text-based role-playing
game in the style of video games popular in the 1960s and 1970s.

We deliberated openly and with goodwill and decided that a text-based video game
would be both interesting and engaging to implement, but also small and simple
enough to complete rapidly and to iterate on efficiently. Additionally, the
business case for video games is exceptionally strong. The value of the video
game market reached \$120B in 2018 \citep{superdata} and even a niche product
can tap into this.

Through a short period of further deliberation, we settle on the name of
\texttt{oubliette}, a reference to that type of dungeon and fitting for the
setting of the game.

Our team-based debate provided me with an insight into defining project scope
with regard to time and collective team ability. I found it important to account
for the possibility of failure and changing requirements. A text-based game is
new territory for our team, but we are confident we can implement a minimum
viable product in the time we have.

To flesh out the core concepts of \texttt{oubliette}, our team organised a
brainstorming session. The meeting was enlightening for me, encouraging my
team members and I to develop the confidence to submit our ideas for
consideration and growing our confidence to constructively criticise and evaluate
our colleague's contributions.

This brainstorming meeting afforded me a newfound appreciation for the value
in blue-sky thinking, flying ideas up the flagpole, touching base with
colleagues to get all of our ducks in a row and planning our team's collective
journey as we attempt to develop an integrated and synergistic solution to
meet real-world needs and generate revenue. Having an open and welcoming
discussion is hugely important for fostering team-wide contributions. No ideas
were rejected--some, however, were parked in the corporate lay-by, but we made
sure to keep their engines running.

\section{Weekly stand-up meetings}
\subsection{Week two}
\textbf{Date: 22/01/20} \\ \textbf{Role: engineer/tester}
\begin{itemize}
\item Updated team on the code I've written that provides the skeleton of
\texttt{oubliette}
\item Describe the roadblock of not having a game map to implement
\item Reported the role of engineer/tester that I've assumed
\item Outlined my plans for increasing the robustness of the code
\end{itemize}

\subsection{Week three}
\textbf{Date: 29/01/20} \\ \textbf{Role: engineer/tester}
\begin{itemize}
\item Unfortunately, I was absent from this meeting for personal reasons
\end{itemize}

\section{Summary}

\bibliography{references}

\end{document}