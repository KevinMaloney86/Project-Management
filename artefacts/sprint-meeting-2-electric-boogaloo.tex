\documentclass[a4paper,11pt]{article}
\usepackage[T1]{fontenc}
\usepackage[utf8]{inputenc}
\usepackage{lmodern}

\title{Sprint meeting}
\author{Starship Troopers}
\date{\today}

\begin{document}

\maketitle

\section*{Members present}
\begin{itemize}
\item Joshua Lee (engineer/tester)
\item Eoin Lernihan (engineer/tester)
\item David Newman (product owner)
\item Kevin Maloney (scrum master)
\end{itemize}

\section*{Process}
The sprint planning took place across two in-person team meetings
which occurred on the 26th of February and the 2nd of March. In the
first meeting the team reviewed user stories completed and allocated
the remaining stories for the upcoming sprint. The release plan was
refined with input from the whole team and, in no small part to our
excellent scrum master, the morale among the team was sky high and
we were all confident that the second sprint would progress as well
as our first one.

As with the first sprint, we unanimously re-agreed that circa 25 story
points was eminently doable when taking into account our team's
rapid velocity.

The second meeting was occupied by the drawing up of the release plan
itself. The product owner and the scrum master used a large whiteboard to
perform this process. The team had unanimously agreed that a whiteboard
would provide the most expressive visual imagery to boost our team's
productivity, an idea we borrowed from Uzility\textregistered\ (check
out \texttt{uzility.com} for more fantastic agile ideas).

The product owner and scrum master ordered the chosen user stories by
priority. Each sprint was allocated a mean number of 25 user points, a
number that the team felt confident that we could complete within a
time-boxed two-week sprint.

The second meeting on the 2nd of March was primarily concerned with
a more in-depth co-ordination and alignment of team members than was
possible at our daily scrum meetings. Development on `oubliette` was
progressing well up to this point and it was looking increasingly
likely that the team's goals laid out at the start of the first sprint
were going to be realised, all due to the philosophy outlined on
\texttt{agilemanifesto.org}.

\end{document}